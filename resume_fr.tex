%-------------------------
% Resume in Latex
% Author : Kishonan Pushpanthan
% Based off of: https://github.com/sb2nov/resume
% License : MIT
%------------------------

\documentclass[letterpaper,11pt]{article}

\usepackage{latexsym}
\usepackage[empty]{fullpage}
\usepackage{titlesec}
\usepackage{marvosym}
\usepackage[usenames,dvipsnames]{color}
\usepackage{verbatim}
\usepackage{enumitem}
\usepackage[hidelinks]{hyperref}
\usepackage{fancyhdr}
\usepackage[english]{babel}
\usepackage{tabularx}
\input{glyphtounicode}


%----------FONT OPTIONS----------
% sans-serif
% \usepackage[sfdefault]{FiraSans}
% \usepackage[sfdefault]{roboto}
% \usepackage[sfdefault]{noto-sans}
% \usepackage[default]{sourcesanspro}

% serif
% \usepackage{CormorantGaramond}
% \usepackage{charter}


\pagestyle{fancy}
\fancyhf{} % clear all header and footer fields
\fancyfoot{}
\renewcommand{\headrulewidth}{0pt}
\renewcommand{\footrulewidth}{0pt}

% Adjust margins
\addtolength{\oddsidemargin}{-0.5in}
\addtolength{\evensidemargin}{-0.5in}
\addtolength{\textwidth}{1in}
\addtolength{\topmargin}{-.5in}
\addtolength{\textheight}{1.0in}

\urlstyle{same}

\raggedbottom
\raggedright
\setlength{\tabcolsep}{0in}

% Sections formatting
\titleformat{\section}{
  \vspace{-4pt}\scshape\raggedright\large
}{}{0em}{}[\color{black}\titlerule \vspace{-5pt}]

% Ensure that generate pdf is machine readable/ATS parsable
\pdfgentounicode=1

%-------------------------
% Custom commands
\newcommand{\resumeItem}[1]{
  \item\small{
    {#1 \vspace{-2pt}}
  }
}

\newcommand{\resumeSubheading}[4]{
  \vspace{-2pt}\item
    \begin{tabular*}{0.97\textwidth}[t]{l@{\extracolsep{\fill}}r}
      \textbf{#1} & #2 \\
      \textit{\small#3} & \textit{\small #4} \\
    \end{tabular*}\vspace{-7pt}
}

\newcommand{\resumeSubSubheading}[2]{
    \item
    \begin{tabular*}{0.97\textwidth}{l@{\extracolsep{\fill}}r}
      \textit{\small#1} & \textit{\small #2} \\
    \end{tabular*}\vspace{-7pt}
}

\newcommand{\resumeProjectHeading}[2]{
    \item
    \begin{tabular*}{0.97\textwidth}{l@{\extracolsep{\fill}}r}
      \small#1 & #2 \\
    \end{tabular*}\vspace{-7pt}
}

\newcommand{\resumeSubItem}[1]{\resumeItem{#1}\vspace{-4pt}}

\renewcommand\labelitemii{$\vcenter{\hbox{\tiny$\bullet$}}$}

\newcommand{\resumeSubHeadingListStart}{\begin{itemize}[leftmargin=0.15in, label={}]}
\newcommand{\resumeSubHeadingListEnd}{\end{itemize}}
\newcommand{\resumeItemListStart}{\begin{itemize}}
\newcommand{\resumeItemListEnd}{\end{itemize}\vspace{-5pt}}

%-------------------------------------------
%%%%%%  RESUME STARTS HERE  %%%%%%%%%%%%%%%%%%%%%%%%%%%%

\begin{document}

\begin{center}
    \textbf{\LARGE Kishonan Pushpanathan} \\ \vspace{3pt}
    \textbf{\large Software Developer} \\ \vspace{10pt}
    \small +33 6 12 58 36 70 $|$ \href{mailto:kishopsh@gmail.com}{\underline{kishopsh@gmail.com}} $|$
    \href{https://linkedin.com/in/pkisho}{\underline{linkedin.com/in/pkisho}} $|$
    \href{https://github.com/kyshx}{\underline{github.com/kyshx}} 
\end{center}

%-----------PROFIL-----------
\section{Profil}
\small{
Développeur polyvalent avec plus de 3 ans d'expérience dans la création d'applications web et cloud. Maîtrise de TypeScript, Node.js et Cloud AWS, avec une solide expérience en pratiques DevOps. 
À la recherche de missions freelance stimulantes ou de postes en CDI qui tirent parti de mes compétences en développement full-stack, cloud computing et architecture logicielle.
}

%-----------COMPÉTENCES TECHNIQUES-----------
\section{Compétences Techniques}
 \begin{itemize}[leftmargin=0.15in, label={}]
    \small{\item{
     \textbf{Langages}{: TypeScript, Python, SQL} \\
     \textbf{Cloud}{: Amazon Web Services (Lambda, DynamoDB, S3, SQS, API Gateway, EC2, RDS et autres), IaC (CDK, SST)} \\
     \textbf{Technologies}{: Node.js, Nest.js, Bun, Hono, AdonisJS, Git, Docker, CI/CD (GitHub Actions), React, Prisma} \\
     \textbf{Tooling}{: Monorepo, Vite, Turborepo, Scripts} \\
     \textbf{Databases}{: PostgreSQL, MongoDB} \\
     \textbf{Likes}{: System Design, Software craftsmanship} \\
     
    }}
 \end{itemize}

%-----------FORMATION-----------
\section{Formations}
  \resumeSubHeadingListStart
    \resumeSubheading
      {Epitech, Expertise Informatique et Innovation}{2018 -- 2023}
      {Master of Science - Digital Transformation}{Paris, France}
    \resumeSubheading
      {Ionis STM}{2020 -- 2021}
      {Teaching and coaching certification}{Paris, France}
  \resumeSubHeadingListEnd

%-----------EXPÉRIENCE PROFESSIONNELLE-----------
\section{Expériences Professionnelles}
  \resumeSubHeadingListStart

    \resumeSubheading
      {Développeur Backend}{Sept. 2023 -- Sept. 2024}
      {Koala}{Paris, France}
      \resumeItemListStart
        \resumeItem{Développement de nouvelles fonctionnalités pour un backend Nest.js à fort trafic en utilisant TypeScript et une architecture hexagonale, améliorant les performances du système et la maintenabilité tout en gérant des milliers de requêtes API quotidiennes}
        \resumeItem{Déploiement et gestion d'applications sur le cloud AWS via CDK, utilisation de EC2, Lambda, RDS, Dynamo, SQS et autres}
        \resumeItem{Acquisition d'une bonne maîtrise de la configuration des monorepos à l'aide d'outils tels que Turborepo, améliorant l'organisation du projet et l'efficacité du développement. Approfondissement de la maîtrise des pratiques CI/CD et IaC}
      \resumeItemListEnd

    \resumeSubheading
      {Développeur Full-Stack}{2021 -- 2023}
      {Casap}{Paris, France}
      \resumeItemListStart
      \resumeItem{Création de solutions backend serverless avec AWS (Lambda, API Gateway, DynamoDB) et TypeScript, améliorant la scalabilité et optimisant les coûts d'infrastructure.}
      \resumeItem{Développement d'applications frontend réactives avec React, GraphQL et TypeScript}
      \resumeItem{Gestion de l'infrastructure cloud avec AWS CDK}
    \resumeItemListEnd

    \resumeSubheading
      {Développeur Junior}{2020 -- 2021}
      {Createrocks}{Paris, France}
      \resumeItemListStart
        \resumeItem{Conception et implémentation d'un back-office avec la stack MERN, améliorant l'efficacité administrative}
        \resumeItem{Migration d'une application Cordova vers React Native, résultant en une amélioration significative des performances.}
      \resumeItemListEnd

  \resumeSubHeadingListEnd

%-----------PROJETS-----------
\section{Projets}
    \resumeSubHeadingListStart
      \resumeProjectHeading
          {\textbf{Contributions aux Projets Open-Source} \emph{Hono, Effect, Naive UI} }{}
          \resumeItem{Contribution à des projets open-source en soumettant des pull requests pour résoudre les issues. Engagé activement avec les créateurs de projets et les communautés via Discord et Slack pour proposer des solutions et des améliorations.}
    \resumeSubHeadingListEnd

\end{document}